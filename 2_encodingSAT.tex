% !TEX root = main.tex
\section{Encoding to SAT}
\begin{mytitle}[Why SAT is not enough] Many problems include features beyond propositional logic, like
\begin{itemize}
    \item first-order terms: variables, constants, functions, \ldots
    \item theories: integer arithmetic, sets, floating points, \ldots
    \item quantifiers: forall and exists
\end{itemize}
In most cases, such features are best handled natively with SMT, but in some cases, these features can be encoded into propositional logic. We can then directly use a SAT solver.
\end{mytitle}

\subsection{Definitions}
\begin{mytitle}[Sort] We fix a set of sorts(types) $T_1, T_2, \ldots$ typically including Bool.
\end{mytitle}
\begin{mytitle}[Term variables] For each sort, we fix an alphabet of term variables $x, y, \ldots, x_1, x_2 \ldots$.
\end{mytitle}
\begin{mytitle}[Function symbols] We fix a set of function symbols $f, g, h, \ldots, f_1, f_2, \ldots$, each with a function signature.
\end{mytitle}
\begin{mytitle}[First-order terms] We define first-order terms $t, s::= x\ |\ f(t_1, t_2,\ldots)$.
\end{mytitle}
\begin{mytitle}[Signature] A signature is a set of sorts and function symbols over those sorts.
\end{mytitle}
\begin{mytitle}[First-order assertions] For a given signature, first-order assertions $A$ are defined by $A, B ::= t\ |\ \lnot A\ |\ A\land B\ |\ A\lor B\ |\ A\To B\ |\ A\Leftrightarrow B\ |\ \forall x:T.A\ |\ \exists x: T.A$, where $t$ is of sort Bool. 
\end{mytitle}
\begin{mytitle}[Value in a model] The value of a term $t$ in a model $M$, written $\val{t}{M}$, is defined by $\val{x}{M}=M(x)$ and $\val{f(t_1, t_2, \ldots)}{M}=M(f)(\val{t_1}{M}, \val{t_2}{M}, \ldots)$.
\end{mytitle}
\begin{mytitle}[Satisfaction] A formula $A$ is satisfied by a model $M$, written $M\models A$, is defined by 
\begin{itemize}
    \item $M\models t$ iff $\val{t}{M}=true$
    \item $M\models \lnot A$ iff $M\not\models A$
    \item $M\models \forall x: T.A$ iff $\forall v\in M(T), M[x \to v]\models A$
    \item $M\models \exists x: T.A$ iff $\exists v\in M(T), M[x \to v]\models A$
\end{itemize}
\end{mytitle}
\begin{mytitle}[Interpretation] Sorts and function symbols are either interpreted or uninterpreted. Interpreted symbols have fixed interpretations, independent of the model.For example, Bool is interpreted and the boolean equality "$=$" is interpreted. Models specify the interpretations fro the free, uninterpreted symbols.
\end{mytitle}

\subsection{Ackermannization}
\begin{mytitle}[General encoding recipe] \hfill
\begin{itemize}
    \item Find a representation for the problem states.
    \item Define an encoding from the original problem into this representation.
    \item Identify and represent missing properties, make sure you don't add any properties.
    \item Optionally optimize redundancy in the representation/encoding.
\end{itemize}
\end{mytitle}
\begin{mytitle}[Equality] 
    \begin{mysubtitle} [Signature definition] Consider a signature consisting of only a single uninterpreted sort $T$ plus the interpreted sort Bool with constants $\top$ and $\bot$, plus the interpreted equality on $T$, written $=$. We consider only quantifier-free formulas. 
    \end{mysubtitle}
    \begin{mysubtitle}[Ackermannization] For each application of equality $x=y$ in the original formula, introduce a fresh variable $eq_{x,y}$ of sort Bool and replace all $x=y$ with $eq_{x,y}$. We also have to conjoin reflexivity ($eq_{x,x}, \ldots$), symmetry ($eq_{x,y} \Leftrightarrow eq_{y,x}$) and transitivity ($eq_{x,y} \land eq_{y,z} \To eq_{x,z}$) to the formula then they are equi-satisfiable.
    \end{mysubtitle}
\end{mytitle}
\begin{mytitle}[Equality and uninterpreted functions] 
    \begin{mysubtitle} [Signature definition] Consider a signature consisting of any number of uninterpreted sorts plus Bool with $\top$ and $\bot$, any number of uninterpreted function symbols over these sorts plus the equality functions "$=$" on each uninterpreted sort. Consider only quantifier-free formulas.
    \end{mysubtitle}
    \begin{mysubtitle}[Ackermannization] For each function application $f(x)$ in the original formula with only variables as a parameter, introduce a fresh variable $f_x$ of the same sort as the return sort of $f$ and replace all occurrences of $f(x)$ with $f_x$. Repeat this process until the original formula contains no function symbols. Each time we introduce a variable $f_x$ to replace $f(x)$, for each variable $f_y$ already introduced to replace $f(y)$, we conjoin $x=y \To f_x = f_y$ to the formula, then they are equi-satisfiable.
    \end{mysubtitle}
\end{mytitle}
\begin{mytitle}[Finitely-bounded quantifiers] In general, reasoning about quantifiers needs more than a SAT solver, however finitely-bounded quantifiers can be easily handled. Two examples:
\begin{itemize}
    \item $\forall x: Int.\ 0 \le x \land x \le 2 \To A \equiv 0 \To A \land 1\To A \land 2\To A$
    \item $\exists x: Int.\ 0 \le x \land x \le 2 \land A \equiv (0 \land A) \lor (1\land A) \lor (2\land A)$
\end{itemize}
\end{mytitle}
