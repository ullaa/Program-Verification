%% !TEX TS-program = latex
%%Vorspann von Ulla Aeschbacher%%
%-----------------------------------------------------------------------------------------------------------------------------------------------------------------------
\usepackage[paper=a4paper,left=27mm,right=27mm,top=30mm,bottom=30mm]{geometry} 	                        	%für Seitenränder und vieles weitere

\usepackage{tikz}												                        		            %tikzpicture
\usetikzlibrary{arrows.meta}									                        	              	%arrowheads length and width
\usetikzlibrary{calc}
\usetikzlibrary{decorations.text}
\usetikzlibrary{decorations.pathmorphing}
\usepackage{pgfplots}                                                                                       %for easily adding plots
\pgfplotsset{compat=1.15}

\usepackage{amsmath}									                        		            		%For multiline equations
\usepackage{ifsym}
\usepackage{amssymb}									                         		            		%For mathbb
\usepackage{mathtools}										                        	            		%For matrices and writing over arrows
\usepackage{bm}												                                       		    %To write bold in math mode
\usepackage{extpfeil}										                                      			%For more varieties in arrows 
\usepackage{extarrows}									            				                        %For long text over long arrows
    
\usepackage{amsthm}										            			                            %For not cursive theorems
\usepackage{framed}										            	                        		    %for frames around theorems
\usepackage{thmtools}											                                    		%for easy custom theorems

\usepackage{enumerate}										                                     			%To be able to customize \benum[a)] \benum[**(i)**] etc...
\usepackage{chngcntr}                                                                                       %to enumerate theorems within section,subsection etc.

\usepackage[parfill]{parskip}									                                       		%no indentation for paragraphs

\usepackage{color}											                                    			%to color text
\usepackage{multicol}										                                    			%use as \begin{multicol}
\usepackage{multirow}									                                    				%for multirow/multicolumn in tabular
\usepackage{caption} 									                                    				%captions for pictures in 
\usepackage{newfloat}
    %for defining new floats
 
\usepackage{listings}									                                    				%for code

\usepackage{algorithm}								                                    					%for algorithms
\usepackage{algpseudocode}							                                    					%from algorithmicx, for creating new blocks
\usepackage{float}									                                    					%use [H] after floating environments to fix placement

\clubpenalty=10000								                                    						%für Hurenkinder und Schusterjungen	
\widowpenalty=10000
%-----------------------------------------------------------------------------------------------------------------------------------------------------------------------
\definecolor{mred}{rgb}{0.82, 0.1, 0.26}
\definecolor{mblue}{rgb}{0.0, 0.73, 0.84}
\definecolor{mgreen}{rgb}{0.0, 0.8, 0.0}
\newcommand{\tr}[1]{\textcolor{mred}{#1}}
\newcommand{\tb}[1]{\textcolor{mblue}{#1}}
\newcommand{\tg}[1]{\textcolor{mgreen}{#1}}
%-----------------------------------------------------------------------------------------------------------------------------------------------------------------------
%   \begin{mydef}[Degree] The number of neigbors of a vertex. \end{mydef}
%	Definition 1 Degree: The number of neigbors of a vertex.
\makeatletter
\declaretheoremstyle[
	spaceabove=6pt, spacebelow=6pt,
	headfont=\normalfont\bfseries,
	notefont=\color{mgreen}\bfseries, 
	notebraces={}{},
	bodyfont=\normalfont,
	postheadspace=0.5 em,
	headpunct={:},
	spacebelow=\parsep,
    	spaceabove=\parsep,
	postheadhook={%
		\ifx\@empty\thmt@shortoptarg
		\renewcommand\addcontentsline[3]{}
		\fi},
]{mydef}
\declaretheorem[name={Definition}, style=mydef]{mydef}
\makeatother

%   \begin{mytitle}[Title] This is the text. \end{mytitle}
%	1 Title: This is the text.
\makeatletter
\declaretheoremstyle[
	spaceabove=6pt, spacebelow=6pt,
	headfont=\normalfont\bfseries,
	notefont=\color{black}\bfseries, 
	notebraces={}{},
	bodyfont=\normalfont,
	postheadspace=0.5 em,
	headpunct={:},
	spacebelow=\parsep,
    	spaceabove=\parsep,
	postheadhook={%
		\ifx\@empty\thmt@shortoptarg
		\renewcommand\addcontentsline[3]{}
		\fi},
	numberwithin=subsection,
]{mytitle}
\declaretheorem[name={}, style=mytitle]{mytitle}
\makeatother

%   \begin{mysubtitle}[Subtitle] This is the text. \end{mysubtitle}
%	1 Subtitle: This is the text.
\makeatletter
\declaretheoremstyle[
	spaceabove=6pt, spacebelow=6pt,
	headfont=\normalfont\bfseries,
	headindent=2em,
	notefont=\color{black}\bfseries, 
	notebraces={}{},
	bodyfont=\normalfont,
	postheadspace=0.5 em,
	headpunct={:},
	spacebelow=\parsep,
    	spaceabove=\parsep,
	postheadhook={%
		\ifx\@empty\thmt@shortoptarg
		\renewcommand\addcontentsline[3]{}
		\fi},
	numberwithin=mytitle,
]{mysubtitle}
\declaretheorem[name={}, style=mysubtitle]{mysubtitle}
\makeatother

%   \begin{mythm}[Gauss] The number of neigbors of a vertex. \end{mythm}
%	\box{Theorem 1 (Gauss): The number of neigbors of a vertex.}
\makeatletter
\declaretheoremstyle[
	spaceabove=6pt, spacebelow=6pt,
	headfont=\normalfont\bfseries,
	notefont=\color{mred}\bfseries, 
	notebraces={}{},
	bodyfont=\normalfont,
	postheadspace=0.5 em,
	headpunct={:},
	spacebelow=\parsep,
    	spaceabove=\parsep,
	postheadhook={%
		\ifx\@empty\thmt@shortoptarg
		\renewcommand\addcontentsline[3]{}
		\fi},
	mdframed={
		linecolor=black, 
		innertopmargin=6pt,
		innerbottommargin=6pt, 
		skipabove=\parsep, 
		skipbelow=\parsep } 
]{mythm}
\declaretheorem[name={Theorem}, style=mythm]{mythm}
\makeatother

%	\bem{The number of neigbors of a vertex.}
%	Remark 1: \italic{The number of neigbors of a vertex.}
\newtheoremstyle{bemdef}{1em}{1em}{\itshape}{}{\bfseries}{:}{0.5em}{}
\theoremstyle{bemdef}
\newtheorem{bemerkung}{Remark}
\newcommand{\bem}[1]{\begin{bemerkung}#1\end{bemerkung}}
%-----------------------------------------------------------------------------------------------------------------------------------------------------------------------
\newcommand\Tstrut{\rule{0pt}{2.6ex}}        									                            % = `top' strut = Abstand vor /hline in array und tabular
\newcommand\Bstrut{\rule[-0.9ex]{0pt}{0pt}}   									                            % = `bottom' strut = Abstand nach /hline in array und tabular

\iffalse
%use like this
\begin{tabular}{l l}
A & B\\
\hline
a & b\Tstrut\\[0.5em]
c & d\\[0.5em]
e & f
\end{tabular}
\fi
%-----------------------------------------------------------------------------------------------------------------------------------------------------------------------
\newcommand{\vekt}[2]{\left(\begin{array}{r r} #1 \\ #2 \end{array}\right)}						            %2-dim Vektor
\newcommand{\vekts}[2]{\left(\begin{smallmatrix} #1\\ #2 \end{smallmatrix}\right)}				            %small Vektor für in-text
\newcommand{\vek}[3]{\left(\begin{array}{r r r} #1 \\ #2 \\ #3 \end{array}\right)}				        	%3-dim Vektor
\newcommand{\vekb}[3]{\left(\begin{array}{r r r} #1 \\[0.3em] #2 \\[0.3em] #3 \end{array}\right)}		    %3-dim Vektor für Brüche
\newcommand{\vekc}[3]{\left(\begin{array}{c c c} #1 \\ #2 \\ #3 \end{array}\right)}                         %zentrierter 3-dim Vektor
\newcommand{\vekbc}[3]{\left(\begin{array}{c c c} #1 \\[0.8em] #2 \\[0.8em] #3 \end{array}\right)}	        %zentrierter 3-dim Vektor für Brüche
\newcommand{\veks}[3]{\left(\begin{smallmatrix} #1\\ #2 \\ #3 \end{smallmatrix}\right)}			            %small Vektor für in-text
\newcommand{\vekk}[4]{\left(\begin{array}{r r r r} #1 \\ #2 \\ #3 \\ #4 \end{array}\right)}			        %4-dim Vektor
\newcommand{\vekks}[4]{\left(\begin{smallmatrix} #1\\ #2 \\ #3 \\ #4 \end{smallmatrix}\right)}		        %small Vektor für in-text
\newcommand{\efrac}[1]{\frac{1}{#1}}												                        %1/x Bruch
%-----------------------------------------------------------------------------------------------------------------------------------------------------------------------
\newcommand{\doubleu}[1]{\underline{\underline{#1}}}							                            %to double underline something in math mode
%-----------------------------------------------------------------------------------------------------------------------------------------------------------------------
\newcommand{\mc}[1]{\mathcal{#1}}											                                %Mathcal

\newcommand{\mb}[1]{\mathbb{#1}}											                                %Mathbb

\newcommand{\ttt}[1]{\texttt{#1}}											                                %Texttt
%-----------------------------------------------------------------------------------------------------------------------------------------------------------------------
\newcommand{\spr}[2]{\left\langle #1,#2 \right\rangle}							                            %Skalarprodukt
\newcommand{\norm}[1]{\left\Vert #1 \right\Vert}								                            %Norm
\newcommand{\abs}[1]{\left\vert #1 \right\vert}									                            %Betrag
%-----------------------------------------------------------------------------------------------------------------------------------------------------------------------
\newcommand{\rank}{\text{rank}}											                                    %rank
\newcommand{\spann}{\text{span}}											                                %span
\newcommand{\im}{\text{Im}}												                                    %Im
\newcommand{\sign}{\text{sign}}										                                    	%sign
\newcommand{\id}{\text{id }}												                                %id
\newcommand{\diag}{\text{diag}}											                                    %diag
\renewcommand{\ker}{\text{ker}}											                                    %ker
\renewcommand{\dim}{\text{dim}}											                                    %dim
%-----------------------------------------------------------------------------------------------------------------------------------------------------------------------
